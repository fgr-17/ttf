% Chapter Template

\chapter{Conclusiones} % Main chapter title

\label{Chapter5} % Change X to a consecutive number; for referencing this chapter elsewhere, use \ref{ChapterX}


%----------------------------------------------------------------------------------------

%----------------------------------------------------------------------------------------
%	SECTION 1
%----------------------------------------------------------------------------------------

\section{Conclusiones generales }

Para este proyecto se utilizaron en forma intensiva los contenidos y herramientas de la Carrera de Especialización en Sistemas Embebidos (CESE). Se pusieron en práctica técnicas de Gestión de Proyectos, documentación manual y automática del trabajo, sistema de versionado de software y hardware y conceptos fundamentales de Ingeniería de Software. Se emplearon conocimientos específicos sobre arquitectura del microcontrolador, modelos de programación, sistema operativo de tiempo real freeRTOS, protocolos de comunicación (BLE, SPI, USB, y de alto nivel) y testing unitarios, entre otros.

El trabajo se considera satisfactorio ya que se abarcaron gran parte de los objetivos, no solo funcionales, sino también académicos y de experiencia. Puntualmente se puede mencionar:

\begin{itemize}

\item Se logró un equipo funcional que cumple con gran parte de los requerimientos iniciales.

\item Se trabajó de una manera escalable, tomando criterios de calidad de software que permiten mantener y ampliar a futuro el firmware.

\item Se generó una apropiada documentación del trabajo, con la posibilidad de involucrar a estudiantes de la Universidad Favaloro para futuras mejoras como parte de su proyecto final.

\item El hardware se diseñó modularmente siguiendo conceptos aprendidos en el CESE, que permiten realizar nuevas versiones, correcciones y mejoras a futuro de una manera profesional.

\item Se utilizaron conceptos de gestión de proyectos que posiblitaron tener una visión más amplia y una experiencia muy rica, que da a lugar a extender las aplicaciones y alcances del dispositivo desarrollado a otras áreas de la ingeniería.

\item Es importante destacar que este proyecto tiene un adoptante concreto que utilizará el equipo para realizar investigaciones asociadas a la ingeniería biomédica. Es decir, es un proyecto de transferencia para equipar a un laboratorio de bioingeniería que planea utilizar el dispositivo para realizar mediciones en animales y desarrollar nuevos biomarcadores vasculares.

\end{itemize}

%----------------------------------------------------------------------------------------
%	SECTION 2
%----------------------------------------------------------------------------------------
\section{Próximos pasos}

Las instancias próximas a resolver son:

\begin{itemize}

\item Diseñar un gabinete a medida.

\item Probar el equipo en campo realizando varias experiencias.

\item Generar un software para interfaz Bluetooth y de PC.

\item Desarrollar el software de procesamiento y eventualmente generar índices de calidad que aseguren que las señales tengan una relación señal/ruido adecuada.

\end{itemize}


Además se planifican las siguientes mejoras a futuro relativas a:

\begin{itemize}

\item Agregar conectividad wifi para acceder al dispositivo remotamente.

\item Mejorar el consumo para poder realizar experiencias más prolongadas.

\item Reemplazar el conversor analógico digital ADS1292 por uno menos sofisticado que sea de menor precio.

\item Agregar entradas adicionales para medir otras variables de interés.

\end{itemize}