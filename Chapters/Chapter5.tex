% Chapter Template

\chapter{Conclusiones} % Main chapter title

\label{Chapter5} % Change X to a consecutive number; for referencing this chapter elsewhere, use \ref{ChapterX}


%----------------------------------------------------------------------------------------

%----------------------------------------------------------------------------------------
%	SECTION 1
%----------------------------------------------------------------------------------------

\section{Conclusiones generales }

La idea de esta sección es resaltar cuáles son los principales aportes del trabajo realizado y cómo se podría continuar. Debe ser especialmente breve y concisa. Es buena idea usar un listado para enumerar los logros obtenidos.

Para este proyecto se utilizaron en forma intensiva la mayoría de los contenidos y herramientas vistas durante el Carrera de Especialización en Sistemas Embebidos (CESE). Se pusieron en práctica técnicas de Gestión de Proyectos, documentación manual y automática del trabajo, sistema de versionado de software y hardware. En cuanto a lo técnico se emplearon conocimientos específicos sobre arquitectura del microcontrolador, modelos de programación, sistema operativo de tiempo real freeRTOS, protocolos de comunicación (BLE, SPI, USB, y de alto nivel), testing unitarios, etc.


%----------------------------------------------------------------------------------------
%	SECTION 2
%----------------------------------------------------------------------------------------
\section{Próximos pasos}

Acá se indica cómo se podría continuar el trabajo más adelante.
